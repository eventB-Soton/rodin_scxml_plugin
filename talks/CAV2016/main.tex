

%-----------------------------------------------------------
\documentclass[10pt,xcolor=svgnames,compress,blackandwhite]{beamer}
\usepackage{etex}
\mode<presentation>
%----------------------------------------------------------
%	colors and appearance
\usetheme{Warsaw}

\setbeamercolor{structure}{fg=SteelBlue!60!Black}
%{fg=SteelBlue!65!black}
\useoutertheme[subsection=false]{miniframes} 
%\useoutertheme{infolines} 
%------------------------------
%
%	block and items
\setbeamertemplate{blocks}[rounded][shadow=true] 
\setbeamertemplate{navigation symbols}{} 
\useinnertheme{circles} 
\setbeamercovered{transparent}
%
%
%\Large size for frame titles
\setbeamercolor{title}{fg=white, bg=SteelBlue!50!black}
\setbeamerfont{frametitle}{size={\large}} 
%\setbeamercolor{frametitle}{fg=black}
\setbeamercolor{frametitle}{bg=SlateGray}
%-----------------------------------------

%------------------------------------------
%	to have the background black.... 
%\setbeamercolor{structure}{fg=red!55!black}
%\beamersetaveragebackground{gray!60!black}
%\setbeamercolor{normal text}{fg=white}
%\setbeamercolor{itemize item}{fg=red!60!black}
%\setbeamercolor{enumerate item}{fg=black}
%------------------------------------------


\usepackage[english]{babel}
\usepackage[latin1]{inputenc}
\usepackage{tikz}
\usepackage{pgf}
\usepackage{verbatim}
\usetikzlibrary{arrows,shapes,backgrounds}
\usepackage{times}
\usepackage[T1]{fontenc}
\usepackage{multimedia}
\usepackage{hyperref}
\usepackage{xkeyval}
\usepackage{grffile} % To parse file names properly in \includegraphics
\usepackage{graphicx} 
\usepackage{subfigure}
\usepackage{xcolor}
\usepackage{rotating}
\usepackage{amsmath,amsfonts,amssymb,amsthm,amsbsy}
% \usepackage{footbib}
\usepackage{bbding}
\usepackage{animate}
\usepackage{multirow}
\usepackage{multicol}
\usepackage{booktabs} 
\usepackage{bm}
\usepackage{xspace}
\usepackage{mathtools}
%\usepackage[font=Times,timeinterval=20, timeduration=45, timewarningfirst=19, colorwarningfirst=green,
%timewarningsecond=68, colorwarningsecond=yellow,timedeath=0]{tdclock}
\usepackage{cite}
\RequirePackage{cite}
\usepackage{anyfontsize}
\usepackage{t1enc}

\usepackage[font=Times,timeinterval=15, timeduration=25, 
timewarningfirst=35, colorwarningfirst=green,
timewarningsecond=70,colorwarningsecond=yellow,
timedeath=0,font=Helv]{tdclock} 
% timewarningfirst=26, colorwarningfirst=green,			
% % end of introduction
% timewarningsecond=74,colorwarningsecond=yellow,
% % end of nanopore
% timedeath=0,font=Helv]{tdclock}


\newcommand{\cgr}[1]{{\color{green} #1}}
\newcommand{\ccy}[1]{{\color{cyan} #1}}
\newcommand{\cye}[1]{{\color{yellow} #1}}
\newcommand{\ssk}{\smallskip}
\newcommand{\bbk}{\bigskip}
\newcommand{\mmk}{\medskip}
%\usepackage{natbib}
%\bibpunct{(}{)}{;}{a}{,}{,} 

\setbeamercolor{lowercol}{fg=black,bg=SteelBlue!22}







 

\input{def}

\usepackage{soul}


\graphicspath{{./figs/}}

% compiles for final build all stuff needed 
% if undefined, then build more rapidly
\def\finalbuild{1}


%%%%%%%%%%%%%%%%%%%%%%%%%%%%%%%%%%%%%%%%%%%%%%%%%%%%%%%%%%%%%%%%%%%%%%%%%%
\title[Resilient Approach for PDE \ \qquad 
Min:~\factorclockfont{1.45} \cronominutes] 
{\fontsize{0.415cm}{0.5cm}\selectfont 
ULFM-MPI Implementation of a Resilient Task-Based Partial Differential Equations Preconditioner}

\author[F.~Rizzi]
% {\small Francesco Rizzi\\ ($8953$)
% \vspace*{-2.5ex}
% }
{\small {\it F.Rizzi}$^\dag$, K.Morris$^\dag$, 
K.Sargsyan$^\dag$, P.Mycek$^\ddag$, C.Safta$^\dag$,\\
O.LeMaitre$^\ddag$, O.Knio$^\ddag$, 
B.Debusschere$^\dag$ 
\vspace*{-2ex}
}

\institute[] % (optional, but mostly needed)
{%{\texttt{\bf fnrizzi@sandia.gov}}\\
$^\dag$Sandia National Laboratories, Livermore, CA\\
$^\ddag$Duke University, Durham, NC
\vspace*{-1ex}
}
% - Use the \inst command only if there are several affiliations.

\date{\scriptsize {\bf FTXS16 Workshop - HPDC16} \\ 
-- May 2016 --
\vspace{-0.25cm} }

% \logo{\includegraphics[height=0.035\textwidth]
% {figs/SNL_Stacked_Black_Blue.pdf}}
%%%%%%%%%%%%%%%%%%%%%%%%%%%%%%%%%%%%%%%%%%%%%%%%%%%%%%%%%%%%%%%%%%%%%%%%%



%%%%%%%%%%%%%%%%%%%%%%%%%%%%%%%%%%%%%%%%%%%%%%%%%%%%%%%%%%%%%%%%%%%%%%%%
\begin{document}


%\tikzstyle{na} = [baseline=-.5ex]
% \tikzstyle{every picture}+=[remember picture]
%
\begin{frame}
\vspace{-0.5cm}
\titlepage

\vspace*{-5mm}
\begin{center}
{\scriptsize
Supported by the US Department of Energy (DOE)\\ 
Advanced Scientific Computing Research (ASCR)\par
}
\end{center}

%\vspace*{1ex}

\vspace{-0.4cm}
\begin{columns}%[c,totalwidth=\textwidth]
\begin{column}{0.7\textwidth}
{\center {\tiny
Sandia National Laboratories is a multi-program 
laboratory managed and operated by
Sandia Corporation, a wholly owned subsidiary 
of Lockheed Martin Corporation, for the
U.S. Department of Energy's National Nuclear 
Security Administration under contract
DE-AC04-94AL85000.\par}}
\end{column}
\end{columns}


\end{frame}


%%%%%%%%%%%%%%%%%%%%%%%%%%%%%%%%%%%%%%%%%%%%%%%%%%%%%%%%%%%%%%%%%%%%%%%%%%%
% \begin{frame}{Outline}
% % \begin{multicols}{2}
% %  \tableofcontents [hideallsubsections]      % [pausesections]
%   \tableofcontents
% %   \end{multicols}
% \end{frame}
% \section{Outline}
% \begin{frame}
% \tableofcontents
% \end{frame}
% %%%%%%%%%%%%%%%%%%%%%%%%%%%%%%%%%%%%%%%%%%%%%%%%%%%%%%%%%%%%%%%%%%%%%%%


\section{Motivation}
\graphicspath{{./figs/}}
% !TEX root = ../SCXMLREF.tex

\section{Introduction}
\label{sec:introduction}

Previous workshop paper HCCV 2016~\cite{Morris_2016}
\ColinCommented{MOVE this citation somewhere}

Formal verification of high-consequence systems requires the analysis
of formal models that capture the properties and functionality of the
system of interest. Although high-consequence controls and systems are
designed to limit complexity, the requirements and consequent proof
obligations tend to increase the complexity of the formal verification.  
Proof obligations for such requirements can be made more tractable using
abstraction/refinement, providing a natural divide and conquer
strategy for controlling complexity.

\Statecharts~\cite{Harel} are often used for high-consequence controls
and other critical systems to provide an unambiguous, executable way
of specifying functional as well as safety, security, and reliability
properties.  While functional properties (usually) can be tested,
safety security and reliability properties (usually) must be proved
formally.  Here we give a binding from \Statecharts to \EventB so that
this type of reasoning can be carried out.  Moreover, hierarchical
encapsulation maps well onto \Statecharts in a way that is not very
different from previous work in \iUMLB~\cite{Snook2006,snook14:_b_statem,Snook12:FMCO}, a diagramatic modelling notation for \EventB.
Binding \iUMLB to a UML version of \Statecharts is natural and the
addition of run-to-completion semantics, expected by \Statechart
designers, is much of the contribution of this work.  Another
contribution is the augmentation of the textual and parse-able format
for \Statecharts, \SCXML to accomodate elements necessary to support formal
analysis. 

While \Statecharts and various semantic interpretations of
\Statecharts admit refinement reified as both hierarchical or parallel
composition (e.g. see Argos~\cite{Maraninchi91theargos}), here, as
previously\cite{snook14:_b_statem}, we focus only on hierarchical
refinement, the form that \EventB natively admits.  Here we define
hierarchical composition to mean nesting new transition systems inside
previously pure states, and parallel composition to be the combination
in one machine of formerly separate transition systems.
A hierarchical development of a system model uses refinement
concepts to link the different levels of abstraction. Each subsequent
level increases model complexity by adding details in the form of
functionality and implementation method. As the model complexity
increases in each refinement level, tractability of the detailed model
can be improved by the use of a graphical representation, with rich
semantics that can support an infrastructure for formal verification.


The semantics adopted here adheres closely to UML \Statecharts~\cite{Alexandre} and is implemented in \iUMLB.
 Models described in \Statecharts are expressed in \SCXML and
 translated into \EventB logic which uses the \Rodin~\cite{Abrial} for
 machine proofs.  UML \Statechart semantics are not the only formal
 semantics that can be bound to the \Statechart graphical
 language~\cite{Eshuis_2009}.  In Statecharts every triggering signal
 can cause transitions that emit other triggers in a cascade.
 Different semantic interpretations of Statecharts resolve these
 cascades differently.  Argos~\cite{Maraninchi91theargos}, for
 example, views cascading transitions as instantaneous and
 simultaneous rather than the queue-based semantics adopted here.
%
The \EventB language
\cite{abrial10:_model_event_b} provides the logic and refinement
theory required to formally analyze a system model.  The open-source
\Rodin \cite{abrial10:_rodin} provides support for \EventB
including automatic theorem provers.  \iUMLB
augments the \EventB language with a graphical interface including
state-machines.  

With suitable restrictions, \Statecharts already provide a sound,
intuitive, visual metaphor for refinement. Outfitted with a formal
semantics, this work borrows from well-used \Statechart practices in
digital design.  The goal of the present work is to provide usable,
well-founded tools that are familiar to designers of high-consequence
systems and yet provide the currently lacking formal guarantees needed
to ensure safety, security, and reliability.

We previously reported~\cite{Morris_2016} our early attempts to relate \Statecharts to \EventB. At that stage we had tried some aspects of the translation by using simplifications and we were beginning to gain insights into the problem, but had not arrived at the translation we now use.

The rest of the paper is structured as follows.  Section~\ref{sec:background} provides background information on \SCXML, \EventB, and \iUMLB.  Section~\ref{sec:secbot} presents the \IDS that is used as our running example.  Section~\ref{sec:discussion} discusses the various challenges for introducing refinement notion into \SCXML and our approach.  Section~\ref{sec:extensions} shows our extensions to \SCXML which are necessary for reasoning about properties of the \SCXML models.  In Section~\ref{sec:translation}, we illustrate our translation \SCXML models into \EventB using the \IDS example.  Section~\ref{sec:example} shows how properties of the \SCXML models can be specified as invariants and verified in \EventB.  We summarise our contribution and conclude in Section~\ref{sec:conclusion}.

%%% Local Variables: 
%%% mode: latex
%%% TeX-master: "../SCXMLREF.tex"
%%% End: 


\section{Algorithm}
\graphicspath{{./figsAlgo/}}
\input{algorithm.tex}

\section{Implementation}
\graphicspath{{./figsImpl/}}
\input{impl.tex}

\section{Results}
\graphicspath{{./figsResults/}}



\begin{frame}

\vspace{0.5cm}
\centering
\begin{block}{}
\centering
\textbf{\Large{{\it {Resilience Results}}}}
\end{block}
%
\end{frame}



\begin{frame}
\frametitle{Test Problem}
%
\vspace{-0.05cm}

\begin{columns}
\begin{column}{0.65\textwidth}
\bi
\item 2D linear elliptic equation.
\item $201^2$ grid, $3x3$ subdomains.
\item Nominally: $3249$ sampling and $2136$ regression tasks. 
\item 1 {\bf \textcolor{green}{server}}, 
14 {\bf \textcolor{pink}{clients}} size 2. 
\item Faults affect clients only.
\ei
\end{column}
\begin{column}{0.4\textwidth}
\vspace{-0.35cm}
\begin{center}
\includegraphics[width=1\textwidth]{pdeSC15Solution.png}\\
{\small True Solution}
\end{center}
\end{column}
\end{columns}
%
\vspace{0.2cm}
%
\begin{columns}
\begin{column}{0.55\textwidth}
\begin{center}
\vspace{-0.4cm}
\includegraphics[width=0.6\textwidth]{/layout/layout.png}\\
{\small Partitioning}
\end{center}
\end{column}
\begin{column}{0.55\textwidth}
\begin{center}
\vspace{-0.5cm}
\includegraphics[width=0.5\textwidth]{SCresil.pdf}\\
{\small SC configuration}
\end{center}
\end{column}
\end{columns}
%
\end{frame}




\begin{frame}
\frametitle{Injecting SDC and Hard Faults}
%
\begin{columns}
\begin{column}{1.1\textwidth}

\begin{block}{Selective reliability}
Inject/perturb applications at target points and evaluate how it behaves. 
Some parts of the algorithm are assumed to be handled in a 
more reliable manner than others. \footnotesize{M.Hoemmen,M.Heroux,2012}
\end{block}

\pause
\begin{block}{Silent Data Corruptions (SDC)}
\bi
% \item SDC: transient and do not cause the termination of the application.
\item Selective reliability: only affect sampling stage.
\item Injection at random; $\# faults = 0.25, 0.5, 1 \%$ of tasks.
\item Corrupt all boundary conditions data of a task.
\item Bit-flip model: random bit-flip in binary representation. 
\ei
\end{block}
%
\pause
\begin{block}{Hard Faults}
\bi
% \item Permanent, cause the termination of the application.
\item Selective reliability: can affect sampling and regression.
\item Injection at random; $2, 4, 6$ clients crashing.
\item Actually kill the processes associated with those ranks.
\ei
\end{block}
\end{column}
\end{columns}
%
\end{frame}






\begin{frame}
\frametitle{Details}
%
\begin{columns}
\begin{column}{1.1\textwidth}
\begin{block}{Silent Data Corruptions (SDC)}
\bi
\item Resilience condition: out of the samples used in the regression, 
the number of uncorrupted samples has to be greater than 
the minimum set needed to have a well-posed 
regression problem. 
% \item Filter $(-100,100)$ to eliminate outrageous data (expert opinion).
\ei
\end{block}
%
\begin{block}{Hard Faults}
\bi
\item Server continues the execution using 
only the clients that are alive.
\item No need for ULFM collectives to rebuild broken communicators.
\ei
\end{block}
\begin{block}{Oversampling}
\bi
\item Oversampling: $\rho >1$, such that $N = \rho N_{nom}^{s}$.
\item $N_{nom}^{s}$: number of samples for the fault-free scenario.
\ei
\end{block}

\end{column}
\end{columns}
%

\begin{columns}
\begin{column}{1.1\textwidth}
\bi
\item Analyze hard faults only.
\item Hard and soft faults together.
\ei
\end{column}
\end{columns}
%
\end{frame}






\begin{frame}
\frametitle{Hard Faults Only}
%
\begin{columns}
\begin{column}{0.6\textwidth}
\bi
% \item clients workload
% \item Hard faults randomly happening during the sampling. 
\item Angular direction $=$ client name. 
\item Data $=$ total number of tasks being handled during the simulation.
\item No-fault case: \\workload is fairly uniform 
% \item Asymmetry becomes increasingly more evident. 
\item As expected, increasing the number of faults causes 
the clients that are alive to handle more and more 
tasks to compensate for those that are dead.
\ei
\end{column}
%
\begin{column}{0.5\textwidth}
\begin{center}
\includegraphics[width=1\textwidth]{clients_load-crop.pdf}\\
{Clients Workload.}
\end{center}
\end{column}
\end{columns}
%
\end{frame}






\begin{frame}
\frametitle{Hard Faults (HF) Only}
%
\begin{columns}
\hspace{-1cm}
\begin{column}{0.33\textwidth}
\begin{center}
{\bf HF Sampling Only}\\
\vspace{0.15cm}
\includegraphics[width=1.28\textwidth]{H_sampling-crop.pdf}
\end{center}
\end{column}
%
\begin{column}{0.33\textwidth}
\begin{center}
{\bf HF Regression Only }\\
\vspace{0.15cm}
\includegraphics[width=1.3\textwidth]{H_regression-crop.pdf}
\end{center}
\end{column}
%
\begin{column}{0.33\textwidth}
\begin{center}
{\bf HF Samp/Regress}\\
\vspace{0.15cm}
\includegraphics[width=1.3\textwidth]{H_both-crop.pdf}
\end{center}
\end{column}
\end{columns}
%
\vspace{0.1cm}
%
\begin{columns}
\begin{column}{1.1\textwidth}
\bi
\item the best case scenario is when all faults affect regression 
because full computational power is available for a longer part of the simulation
\item HF for both: losing $14 \%$, $28 \%$ and $42 \%$ 
of the clients yields, respectively, a total overhead 
of $8 \%$, $19 \%$ and $30 \%$. 
\ei
\end{column}
\end{columns}
%
\end{frame}




\begin{frame}
\frametitle{Hard Faults and SDC}
%
\begin{columns}
\hspace{-1cm}

\begin{column}{0.33\textwidth}
\begin{center}
{\bf 0.25 \%~SDC}\\
\vspace{0.15cm}
\includegraphics[width=1.26\textwidth]{H+Ssoft025-crop.pdf}
\end{center}
\end{column}
%
\begin{column}{0.33\textwidth}
\begin{center}
{\bf 0.5 \%~SDC}\\
\vspace{0.15cm}
\includegraphics[width=1.26\textwidth]{H+Ssoft050-crop.pdf}
\end{center}
\end{column}
%
\begin{column}{0.33\textwidth}
\begin{center}
{\bf 1.0~\%~SDC}\\
\vspace{0.15cm}
\includegraphics[width=1.26\textwidth]{H+Ssoft100-crop.pdf}
\end{center}
\end{column}
\end{columns}
%
\vspace{0.1cm}
%
\begin{columns}
\begin{column}{1.1\textwidth}
\bi
\item Consider $4$ hard faults; \textit{four-fold} increase in 
SDC from $9$ to $33$ causes 
the sampling overhead to only increase from $9 \%$ to about $15 \%$. 
\item Regression overhead only increases from $30 \%$ to about $38 \%$. 
\item This yields the total overhead to 
only increase from $21 \%$ to $28 \%$. 
\ei
\end{column}
\end{columns}
%
\end{frame}





% \begin{frame}
% \frametitle{Results: $7 \%$ Oversampling}
% %
% \begin{columns}
% \begin{column}{0.55\textwidth}
% \begin{center}
% {\bf All-Data Corruption}\\
% \vspace{0.15cm}
% \includegraphics[width=1\textwidth]
% {SamplingFaultsOnly_corruptAllDataSmallRegTol_7percentOversampling-crop.png}
% \end{center}
% \end{column}
% \begin{column}{0.55\textwidth}
% \begin{center}
% {\bf Single Corruption}\\
% \vspace{0.15cm}
% \includegraphics[width=1\textwidth]
% {SamplingFaultsOnly_corruptSingleDataSmallRegTol_7percentOversampling-crop.png}
% \end{center}
% \end{column}
% \end{columns}
% %
% \vspace{0.1cm}
% %
% \begin{columns}
% \begin{column}{1.1\textwidth}
% \bi
% \item Less oversampling yields smaller overehead. 
% \item Overhead is smaller when all the data in a sampling task is corrupted. 
% \item Note: four-fold faults increase $\Rightarrow$ minimal change in the overhead.
% \ei
% \end{column}
% \end{columns}
% %
% % %
% \end{frame}







% \begin{frame}
% \frametitle{Scaling}
% %
% \begin{columns}
% \begin{column}{1.0\textwidth}
% \bi
% \item Edison (NERSC), using native Cray-MPICH.
% \item Elliptic PDE on unit square. 
% \item Fix the number of clients per server and 
% the amount of data owned by each server, while 
% increasing the problem size 
% by adding increasingly more clusters. 

% %N subdomains: $12^2,24^2,48^2,96^2$.
% % \item N cores: $144$, $576$, $2304$, $9216$.
% % \item Grid/subdomain: $\sim 100^2$.
% \ei
% \end{column}
% \end{columns}
% %
% \begin{columns}
% \begin{column}{0.55\textwidth}
% \bi
% \item Subdomains: $12^2,18^2,24^2,30^2,36^2,42^2$.
% \item Sub grid size: $180^2$.
% \item N servers: $16,36,64,100,144,196$.
% \item Num clients/server: $64$.
% \item Size of client: $4$ MPI ranks.
% \ei
% % \begin{center}
% % \includegraphics[width=1\textwidth]{weak}\\
% % {\bf Weak Scaling}
% % \end{center}
% \end{column}
% %
% \begin{column}{0.55\textwidth}
% % \bi
% % \item N subdomains: $12^2,24^2,48^2,96^2$.
% % \item N cores: $144$, $576$, $2304$, $9216$.
% % \item Full Grid: $2501^2$.
% % \ei
% \begin{center}
% \includegraphics[width=1\textwidth]{weak-eps-converted-to.pdf}\\
% {\bf Weak Scaling}
% \end{center}
% \end{column}
% \end{columns}
% %
% \end{frame}




\section{Conclusions}


\begin{frame}
\frametitle{Conclusions and Ongoing Work}
%
\begin{columns}
% \hspace{-0.7cm}
\begin{column}{1.1\textwidth}
\bi
\item Application is resilient to:
  \bi
  \item Silent Data Corruptions during sampling.
  \item Missing data due to communication issues or node failures.
  \ei
  \mmk
\item Sampling/decomposition approach provides concurrency/parallelism.
\mmk
\item Convergence is achieved in all cases.
\mmk
\item Scalability is excellent.
\mmk
\item Ongoing work/outlook:
  \bi
  \item Dimensionality reduction.
  \ssk
  \item Extension to other types of PDE.
  % \ssk
  % \item Other faults?  
  \ei
\ei
%
\end{column}
\end{columns}
%
\end{frame}



%------------------------------



% \begin{frame}
% \frametitle{Overview of Past/Current Research Interests}
% %
% \begin{columns}
% \hspace{-1.6cm}
% \begin{column}{1\textwidth}
% \begin{figure}
% \includegraphics[width=1.15\textwidth]
% {researchOverview.pdf}
% \end{figure}
% \end{column}
% \end{columns}

% %
% % \begin{columns}
% % \begin{column}{1.1\textwidth}
% % \bi
% % \item Uncertainty Quantification: sparse grids, 
% % high-dimensionality, adaptive methods, forward/inverse problems, 
% % Bayesian inference, sensitivity analysis.
% % \mmk
% % \item Reactive nano-materials: diffusion, inverse problems.
% % \mmk
% % \item Molecular Dynamics: heterogenous systems, nanopores, 
% % LAMMPS (Sandia).
% % \mmk
% % \item UQ for Combustion: sensitivity to reactions rates uncertainties.
% % \mmk
% % \item CFD: fluid, mixing, vortex dynamics, Lagrangian methods, 
% % gravity currents, stratified flows.
% % \ei
% % %
% % \end{column}
% % \end{columns}
% %
% \end{frame}



% %------------------------------



% \begin{frame}
% \vspace{0.5cm}
% \begin{center}
% \begin{block}{   }
% \centering
% {\LARGE{{Thanks for your attention!}}}
% \end{block}
% \end{center}
% %
% \begin{columns}
% \begin{column}{0.33\paperwidth}
% \begin{figure}
% \includegraphics[width=0.93\textwidth]
% {./figsAlgo/2dAlgo/2dalgo.pdf}
% \end{figure}
% \end{column}
% %
% \hspace{-0.05cm}
% %
% \begin{column}{0.33\paperwidth}
% \begin{figure}
% \includegraphics[width=0.85\textwidth]
% {./figsImplementation/TMschematic}
% \end{figure}
% \end{column}
% %
% \hspace{-0.05cm}
% %
% \begin{column}{0.33\paperwidth}
% \begin{figure}
% \includegraphics[width=1\textwidth]
% {./figs2D/HF_rmsResidual_vs_numFaults.eps}
% \end{figure}
% \end{column}
% \end{columns}
% %
% % \bbk
% % \mmk
% % \begin{center}
% % \centering
% % Research supported by: \\ 
% % US, DOE - Office of Advanced Scientific Computing Research
% % \end{center}
% \end{frame}
% %----------------------------------



\begin{frame}
\frametitle{Acknowledgments}
%
\begin{columns}
\begin{column}{0.95\textwidth}
\centering
This material is based upon work supported by the 
U.S. Department of Energy, Office of Science, Office of 
Advanced Scientific Computing Research, 
under Award Number 13-016717.\par 
\end{column}
\end{columns}
%
\bbk
\mmk

\begin{columns}
\begin{column}{0.95\textwidth}
\centering
Sandia National Laboratories is a multi-program 
laboratory managed and operated by Sandia Corporation, 
a wholly owned subsidiary of Lockheed Martin Corporation, 
for the U.S. Department of Energy's 
National Nuclear Security Administration under contract 
DE-AC04-94AL85000.\par
\end{column}
\end{columns}
%
\bbk
\mmk

\pause
\begin{center}
{\Large \bf THANK YOU!}
\end{center}

\end{frame}


%%%%%%%%%%%%%%%%%%%%%%%%%%%%%%%%%%%%%%%%%%%%%%%%%%%%%%%%%%%%%%%%%%%%%%%

\end{document}

