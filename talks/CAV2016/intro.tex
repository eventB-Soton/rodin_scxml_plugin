


\begin{frame}{Do We Need Resilience?}
%%%%%%%%%%
\initclock
%%%%%%%%%%

\begin{columns}
\begin{column}{1.05\textwidth}

\vspace{-4.195cm}
\begin{block}{Resilience}
\begin{center}
Collection of techniques to keep applications 
running to a correct solution in a timely and 
efficient manner {\it despite} underlying system faults.
\end{center}
\end{block}
{ \footnotesize Addressing Failures in Exascale Computing, 
Snir {\it et al.}, 2012.\\
Toward Exascale Resilience, Cappello {\it et al.}, 2014.}
\end{column}
\end{columns}
%
\vspace{-0.35cm}
%
\begin{columns}
\hspace{-1cm}
\begin{column}{0.65\textwidth}
\end{column}
\begin{column}{0.4\textwidth}
\end{column}
\end{columns}
%
\end{frame}






\begin{frame}{Do We Need Resilience?}

\begin{columns}
\begin{column}{1.05\textwidth}

% \vspace{-1cm}
\begin{block}{Resilience}
\begin{center}
Collection of techniques to keep applications 
running to a correct solution in a timely and 
efficient manner {\it despite} underlying system faults.
\end{center}
\end{block}
{ \footnotesize Addressing Failures in Exascale Computing, 
Snir {\it et al.}, 2012.\\
Toward Exascale Resilience, Cappello {\it et al.}, 2014.}
\end{column}
\end{columns}
%
\vspace{-0.35cm}
%
\begin{columns}
\hspace{-1cm}
\begin{column}{0.65\textwidth}
\bi
\item[] \underline{IEEE, Feb.2016, Al Geist (ORNL)}
\item[] ``As a child, were you ever afraid that a monster 
lurking in your bedroom would leap out of the dark and get you?''
\mmk
\item[] ``Lack of resilience is a similar monster, hiding 
in the steel cabinets of the supercomputers and threatening 
to crash the largest computing machines.''
\ei
\end{column}
\begin{column}{0.4\textwidth}
\begin{figure}
% \vspace{0.5cm}
\includegraphics[width=0.85\textwidth]{MjcyMTEwMQ.png}
\end{figure}
\end{column}
\end{columns}
%
\end{frame}




% \begin{frame}{Do We Need Resilience?}

% \vspace{-1cm}
% \begin{columns}
% \begin{column}{0.575\textwidth}
% \bi
% \item Yes? No? Maybe?
% \item IEEE, Feb.2016, Al Geist (ORNL).
% \item Many factors affecting resilience. 
% \ei
% \end{column}
% \begin{column}{0.55\textwidth}
% \begin{figure}
% \includegraphics[width=1.0\textwidth]{article}
% \end{figure}
% \end{column}
% \end{columns}
% %
% % \vspace{0.5cm}
% \begin{columns}
% \hspace{-1cm}
% \begin{column}{0.65\textwidth}
% \bi
% \item[] ``As a child, were you ever afraid that a monster 
% lurking in your bedroom would leap out of the dark and get you?''
% \ssk
% \item[] ``Lack of resilience is a similar monster, hiding 
% in the steel cabinets of the supercomputers and threatening 
% to crash the largest computing machines.''
% \ei
% \end{column}
% \begin{column}{0.4\textwidth}
% \begin{figure}
% % \vspace{0.5cm}
% \includegraphics[width=0.8\textwidth]{MjcyMTEwMQ.png}
% \end{figure}
% \end{column}
% \end{columns}
% %
% \end{frame}





\begin{frame}{Terminology}

% \vspace{-0.85cm}
\begin{columns}
\begin{column}{1.1\textwidth}
\bi
\item Fault: the cause of an error 
%(e.g.: bug, stuck bit, alpha particle.)
\bi
\item active/inactive fault cause/not cause errors.
% inactive fault: do not cause errors.
\item generally local to a single component.
\item e.g. a cracked wire inside a cable.
\ei
%
\ssk
\item Error: the part of the state that may 
lead to a failure %(e.g.: bad value)
\bi
\item may propagate from component to component.
\item e.g. incorrect bit flip during transmission caused by wire.
\ei
%
\ssk
\item Failure: transition to incorrect service 
\bi
\item transition from correct service to incorrect service.
\item e.g. incorrect bit may lead to wrong result. 
%the start of an unplanned service outage
\ei
%
\mmk
% \item Degraded mode/partial failure: the failure of a 
% subset of services.
\ei
\end{column}
\end{columns}
%
\begin{columns}
\begin{column}{1.0\textwidth}
\begin{figure}
\vspace{-0.25cm}
\includegraphics[width=0.8\textwidth]{fault.png}\\
{\footnotesize Addressing Failures in Exascale Computing, 
Snir {\it et al.}, 2012.}
\end{figure}
\end{column}
\end{columns}

% \begin{columns}
% \begin{column}{1.1\textwidth}
% \bi
% \item[] Example: consider a cracked wire inside a cable. 
% \item[] The crack is the fault, local to that cable. 
% \item[] Crack causes a certain bit to be incorrectly flipped 
% during transmission $\rightarrow$ error is an incorrect bit value. 
% \item[] The error may continue to propagate from device to device, 
% perhaps leading to incorrect results (a failure). 
% Or that flipped bit may have no effect on final results (no failure).
% \ei
% \end{column}
% \end{columns}

\end{frame}





\begin{frame}{Broadly Speaking}

% \vspace{-0.85cm}
\begin{columns}
\begin{column}{1.1\textwidth}
\bi
\item Hard: activation is systematically reproducible.

\ssk 
\item Soft: activation is not systematically reproducible.

\ssk 
\item Active: fault causes an error.

\ssk 
\item Dormant: fault does not cause an error. \\
The dormant fault is activated when it causes an error. 

\ssk 
\item Permanent: presence is continuous in time.

\ssk 
\item Transient: presence is temporary.

\ssk 
\item Intermittent: fault is transient and reappears.

\bbk
{\footnotesize Addressing Failures in Exascale Computing, 
Snir {\it et al.}, 2012.}
\ei
\end{column}
\end{columns}
%
\end{frame}






% \begin{frame}{Broadly Speaking}

% % \vspace{-0.85cm}
% \begin{columns}
% \begin{column}{1.1\textwidth}
% %Faults and errors are broadly categorized as:
% \ssk

% \bi
% \item 
% \ei


% % \bi
% % \item Hard (persistent):
% % \bi
% % \item hardware component fails, requires 
% % human intervention to be fixed. 
% % \item e.g. a node failing.
% % \ei
% % %
% % %
% % \bbk
% % \item Soft (transient):
% % \ssk
% % \bi
% % \item Detectable: corrected by the hardware 
% % or low-level system software. 
% % \bi
% % \item e.g. single bit-flip correction by Error-Correcting Code (ECC).
% % \ei
% % %
% % \mmk
% % \item Silent: an error occurred 
% % but the hardware and low-level system software 
% % could not detect it.
% % \bi
% % \item e.g. double/triple bit errors.
% % \ei
% % \ei
% % %
% % % \mmk
% % % \item How do we rank them? Which one is worse? \\
% % % Many factors play a role in this but...
% % \ei
% %
% % \bbk
% % \centering
% \end{column}
% \end{columns}
% %
% \end{frame}








% \begin{frame}{``HPC Inferno''}

% \vspace{-0.5cm}
% \begin{columns}
% \begin{column}{0.5\textwidth}
% \begin{figure}
% \includegraphics[width=1.0\textwidth]{dante.jpg}
% \end{figure}
% \end{column}
% %
% \begin{column}{0.5\textwidth}
% \end{column}
% \end{columns}
% %
% \ssk
% \begin{columns}
% \begin{column}{1\textwidth}
% \centering
% \textcolor{white}{( Changes/suggestions are welcome... )}
% \end{column}
% \end{columns}
% %
% \end{frame}




% \begin{frame}{``HPC Inferno''}

% \vspace{-0.5cm}
% \begin{columns}
% \begin{column}{0.5\textwidth}
% \begin{figure}
% \includegraphics[width=1.0\textwidth]{dante.jpg}
% \end{figure}
% \end{column}
% %
% \begin{column}{0.5\textwidth}
% \begin{figure}
% \includegraphics[width=1.0\textwidth]{dante2}
% \end{figure}
% \end{column}
% \end{columns}
% %
% \ssk
% \begin{columns}
% \begin{column}{1\textwidth}
% \centering
% ( Changes/suggestions are welcome... )
% \end{column}
% \end{columns}
% %
% \end{frame}






\begin{frame}{How Many Bit Flips At Petascale?}

\vspace{-2.8cm}
\begin{columns}
\hspace{-1cm}
\begin{column}{1.0\textwidth}
\bi
\item Jaguar: Cray XT5 system at Oak Ridge (until 2012).
%
\ssk
\item Jaguar had 360 terabytes of main memory, 
all protected by ECC.
%
\ssk
\item Guess how often Jaguar had a bit spontaneously change state?
\vspace{-0.3cm}
\ei
%
\end{column}
\end{columns}
%
\end{frame}





\begin{frame}{How Many Bit Flips At Petascale?}

\begin{columns}
\hspace{-1cm}
\begin{column}{1.0\textwidth}
\bi
\item Jaguar: Cray XT5 system at Oak Ridge (until 2012).
%
\ssk
\item Jaguar had 360 terabytes of main memory, 
all protected by ECC.
%
\ssk
\item Guess how often Jaguar had a bit spontaneously change state?
\vspace{-0.3cm}
\begin{center}
\item[] {\large $\sim 300$ per minute!
\footnote{Al Geist, IEEE Spectrum, 2016}
\footnote{A Field Study of DRAM Errors, V.Sridharan, D.Liberty, 2012}
}
\end{center}
%
\ssk
\item At exascale error rates will become larger. 
\bi
\item Hardware failures are expected to be more frequent.
\ssk
\item More complex hardware $\Rightarrow$ more complex, 
error-prone software.
\ssk
\item Application codes are becoming more complex.
\ei
\ei
%
\end{column}
\end{columns}
%
\end{frame}







\begin{frame}{Challenges \& Needs}
%
%
\begin{columns}
%
\begin{column}{1.1\textwidth}
\bi
\item {\it Hypothetically} hardware can take care of most of it...

\bbk
\item ...at the expense of energy consumption, money, 
and asynchrony.

\bbk
\item Power is a big obstacle towards exascale. 

\bbk
\item High tradeoffs between power, resiliency 
and performance. 
\ssk
\item[] For instance: if an application can tolerate memory bit flips for 
certain parts of its memory, it can ask the OS to turn off ECC checks for 
those memory regions and potentially improve power and performance.

\bbk
\item Cross-cutting research needed to explore these areas.
\ei
\end{column}
\end{columns}
%Q
\end{frame}







