% !TEX root = ../SCXMLREF.tex

\section{Extensions to SCXML}
\label{sec:extensions}
 
The following syntax extensions are added to \SCXML models to support modelling features needed in \iUMLB/\EventB. These extensions are prefixed with `iumlb:' in order to distinguish them from the scxml XML parser. (So that they are ignored by \SCXML simulation tools). 
%They are loaded by EMF as generic feature maps (‘Any’ for contained elements and ‘AnyAttribute’ for attributes).
\begin{itemize}
	\item \textbf{iumlb:refinement} - an integer attribute representing the refinement level at which the parent element should be introduced (see Listing~\ref{lst:secBot}, line~\ref{line:refinement}).
	\item \textbf{iumlb:invariant} - an element that generates an invariant in \iUMLB. This provides a way to add invariants to states so that important properties concerning the synchronisation of state with ancilliary data and other state machines can be expressed (see Listing~\ref{lst:secBot}, line~\ref{line:invariant}).
	\item \textbf{iumlb:guard} - an element that generates a transition guard in \iUMLB. 
	This provides a way to add new guard conditions to transitions over several refinement (Listing~\ref{lst:secBot}, line~\ref{line:guard}) as well as providing an element with attributes such as derived (for \EventB theroems), name and comment.
	\item \textbf{iumlb:predicate} - a string attribute used for the predicate of a guard or invariant (Listing~\ref{lst:secBot}, line~\ref{line:predicate}).
	\item \ldots other attributes useful for \iUMLB elements: name, derived, type, comment.
\end{itemize}

\begin{lstlisting}[caption={\textbf{Wait50ms} state snippet of SCXML model representation illustrating the use of different SCXML modeling features, as well as, added syntax extensions},label={lst:secBot}, language=xml, escapechar=|, frame=single]
...
<state id="Wait50ms" iumlb:refinement="2">
	<initial iumlb:refinement="2"> |\label{line:refinement}|
		<transition cond="cnt=0" target="Increment"/> 
	</initial>
	<iumlb:invariant name="check_cnt" iumlb:predicate="cnt &lt; max" iumlb:refinement="2"/> |\label{line:invariant}|
	<transition cond="" target="Go" iumlb:refinement="0" />
	<state id="Increment" iumlb:refinement="2">
		<transition cond="" event="tick" target="Increment">
			<assign attr="cnt" expr="cnt+1" location="cnt"/>
			<iumlb:guard name="stillCounting" iumlb:predicate="cnt &lt; 5"/> |\label{line:predicate}|
		</transition>
		<transition cond="" target="WAITDone">
			<iumlb:guard name="doneCounting" iumlb:predicate="cnt = 5"/> |\label{line:guard}|
    	</transition>
	</state>
	<final id="WAITDone" iumlb:refinement="2"/>
	<datamodel>
		<data expr="0" id="cnt" src="" iumlb:type="NAT "/>
	</datamodel>
</state>
...
\end{lstlisting}


Hierarchical nested state charts are translated into similarly structured \iUMLB state-machines. The generated \iUMLB model contains refinements that add nested state-machines as indicated in the  \SCXML \statechart by the \textbf{iumlb:refinement} attributes annotated on state elements. \iUMLB transitions are generated for each \SCXML transition and linked to \EventB events that represent each of the possible synchronisations that could involve that transition.


%%% Local Variables:
%%% mode: latex
%%% TeX-master: "../SCXMLREF"
%%% End: