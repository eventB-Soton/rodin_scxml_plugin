% !TEX root = ../SCXMLREF.tex

\section{Introduction}
\label{sec:introduction}

Previous workshop paper HCCV 2016~\cite{Morris_2016}
\ColinCommented{MOVE this citation somewhere}

Formal verification of high-consequence systems requires the analysis
of formal models that capture the properties and functionality of the
system of interest. Although high-consequence controls and systems are
designed to limit complexity, the requirements and consequent proof
obligations tend to increase the complexity of the formal verification.  
Proof obligations for such requirements can be made more tractable using
abstraction/refinement, providing a natural divide and conquer
strategy for controlling complexity.

\Statecharts~\cite{Harel} are often used for high-consequence controls
and other critical systems to provide an unambiguous, executable way
of specifying functional as well as safety, security, and reliability
properties.  While functional properties (usually) can be tested,
safety security and reliability properties (usually) must be proved
formally.  Here we give a binding from \Statecharts to \EventB so that
this type of reasoning can be carried out.  Moreover, hierarchical
encapsulation maps well onto \Statecharts in a way that is not very
different from previous work in \iUMLB~\cite{Snook2006,snook14:_b_statem,Snook12:FMCO}, a diagramatic modelling notation for \EventB.
Binding \iUMLB to a UML version of \Statecharts is natural and the
addition of run-to-completion semantics, expected by \Statechart
designers, is much of the contribution of this work.  Another
contribution is the augmentation of the textual and parse-able format
for \Statecharts, \SCXML to accomodate elements necessary to support formal
analysis. 

While \Statecharts and various semantic interpretations of
\Statecharts admit refinement reified as both hierarchical or parallel
composition (e.g. see Argos~\cite{Maraninchi91theargos}), here, as
previously\cite{snook14:_b_statem}, we focus only on hierarchical
refinement, the form that \EventB natively admits.  Here we define
hierarchical composition to mean nesting new transition systems inside
previously pure states, and parallel composition to be the combination
in one machine of formerly separate transition systems.
\ColinCommented{Should we briefly say what parallel refinment is?}
\RobCommented{I hope it is OK to make the heirarchical vs parallel
  composition for refinement.  It leaves open work to be done.  Also,
  I found exactly one reference for iUML-B in the .bib, I think Colin
  and Karla's paper in the NASA conference might also be something to
  cite.}  A hierarchical development of a system model uses refinement
concepts to link the different levels of abstraction. Each subsequent
level increases model complexity by adding details in the form of
functionality and implementation method. As the model complexity
increases in each refinement level, tractability of the detailed model
can be improved by the use of a graphical representation, with rich
semantics that can support an infrastructure for formal verification.


The semantics adopted here adheres closely to UML \Statecharts~\cite{Alexandre} and is implemented in \iUMLB.
%% \ColinCommented{They are somewhat different.. can we cleave to both of them? 
%% 	My view is that we adopt the semantics similar to those of UML and try to represent them in the alternative iUML-B semantics (that cleaves closely to those of Event-B)}
 Models described in \Statecharts are expressed in \SCXML and
 translated into \EventB logic which uses the \Rodin~\cite{Abrial} for
 machine proofs.  UML \Statechart semantics are not the only formal
 semantics that can be bound to the \Statechart graphical
 language~\cite{Eshuis_2009}.  In Statecharts every triggering signal
 can cause transitions that emit other triggers in a cascade.
 Different semantic interpretations of Statecharts resolve these
 cascades differently.  Argos~\cite{Maraninchi91theargos}, for
 example, views cascading transitions as instantaneous and
 simultaneous rather than the queue-based semantics adopted here.
%
The \EventB language
\cite{abrial10:_model_event_b} provides the logic and refinement
theory required to formally analyze a system model.  The open-source
\Rodin \cite{abrial10:_rodin} provides support for \EventB
including automatic theorem provers.  \iUMLB~\cite{snook14:_b_statem}
augments the \EventB language with a graphical interface including
state-machines.  

With suitable restrictions, \Statecharts already provide a sound,
intuitive, visual metaphor for refinement. Outfitted with a formal
semantics, this work borrows from well-used \Statechart practices in
digital design.  The goal of the present work is to provide usable,
well-founded tools that are familiar to designers of high-consequence
systems and yet provide the currently lacking formal guarantees needed
to ensure safety, security, and reliability.

The rest of the paper is structured as follows.  Section~\ref{sec:background} provides background information on \SCXML, \EventB, and \iUMLB.  Section~\ref{sec:secbot} presents the intrusion dection system that is used as our running example.  Section~\ref{sec:discussion} discusses the various challenges for introducing refinement notion into \SCXML and our approach.  Section~\ref{sec:extensions} shows our extensions to \SCXML which are necessary for reasoning about properties of the \SCXML models.  In Section~\ref{sec:translation}, we illustrate our translation \SCXML models into \EventB using the intrusion detection system example.  Section~\ref{sec:example} shows how properties of the \SCXML models can be specified as invariants and verified in \EventB.  We summarise our contribution and conclude in Section~\ref{sec:conclusion}.

%%% Local Variables: 
%%% mode: latex
%%% TeX-master: "../SCXMLREF.tex"
%%% End: 
