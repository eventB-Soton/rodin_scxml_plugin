% !TEX root = ../SCXMLREF.tex

\section{Introduction}
\label{sec:introduction}

Formal verification of high consequence systems requires the analysis
of formal models that capture the properties and functionality of the
system of interest. Although high-consequence controls and systems are
designed to limit complexity, the requirements and consequent proof
obligations tend increase complexity of verification.  Proof
obligations for such requirements can be made more tractable using
abstraction/refinemen, providing a natural divide and conquer
strategy for controlling complexity.

Statecharts\cite{Harel} are often used for high-consequence controls
and other critical systems to provide an unambiguous, executable way
of specifying functional as well as safety, security, and reliability
properties.  While functional properties (usually) can be tested,
safety security and reliability properties (usually) must be proved
formally.  Here we give a binding from Statecharts to Event-B so that
this type of reasoning can be carried out.  Moreover, heirarchical
encapsulation maps well onto Statecharts in a way that is not very
different from previous work in iUML-B\cite{snook14:_b_statem}.
Binding iUML-B to the a UML version of Statecharts is natural and the
addition of run-to-completion semantics, expected by Statechart
designers, is much of the contribution of this work.  Another
contribution is the augmentation of the textual and parse-able format
for Statecharts, SCXML to accomodate elements necessary to support formal
analysis. 

While Statecharts and various semantic interpretations of Statecharts
admit refinement reified as both hierarchical or parallel composition
(e.g. see Argos\cite{Maraninchi91theargos}), here, as
previously\cite{snook14:_b_statem}, we focus only on heirarchical
refinement, the form that Event-B natively admits.
\RobCommented{I hope it is OK to make the heirarchical vs parallel
  composition for refinement.  It leaves open work to be done.  Also,
  I found exactly one reference for iUML-B in the .bib, I think Colin
  and Karla's paper in the NASA conference might also be something to
  cite.}  A hierarchical development of a system model uses refinement
concepts to link the different levels of abstraction. Each subsequent
level increases model complexity by adding details in the form of
functionality and implementation method. As the model complexity
increases in each refinement level, tractability of the detailed model
can be improved by the use of a graphical representation, with rich
semantics that can support an infrastructure for formal verification.

The semantics adopted here cleaves closely to those of iUML-B
specifically and UML Statecharts\cite{Alexandre} generally. Models
decribed in Statecharts are expressed in SCXML and translated into
Event-B logic and uses the Rodin framework\cite{Abrial} for machine
proofs.  UML Statechart semantics are not the only formal semantics that can be bound to the Statechart graphical language\cite{Eshuis_2009}.
%
The Event-B language
\cite{abrial10:_model_event_b} provides the logic and refinement
theory required to formally analyze a system model.  The open-source
Rodin tool \cite{abrial10:_rodin} provides support for Event-B
including automatic theorem provers.  iUML-B \cite{snook14:_b_statem}
augments the Event-B language with a graphical interface including
state-machines.  

With suitable restrictions, Statecharts already provide a sound,
intuitive, visual metaphor for refinement. Outfitted with a formal
semantics, this work borrows from well-used Statechart practices in
digital design.  The goal of the present work is to provide usable,
well-founded tools that are familiar to designers of high-consequence
systems and yet provide the currently lacking formal guarantees needed
to ensure safety, security, and reliability.
