% !TEX root = ../SCXMLREF.tex

\section{Conclusion}
\label{sec:conclusion}
We have shown how a slightly extended and annotated startchart, with a typical 'run to completion' semantic, can be translated into the \EventB notation for verification of synchronisation properties using the powerful \EventB theorem proving tools.
Furthermore, borrowing from the refinement concepts of \EventB, we introduce a notion of refinement to statecharts and demonstrate how the proof of a property at an abstract level, helps formulate constraints that must apply (and will be verifed to do so) in further refinements.

In future work we will continue to experiment with different examples to explore the alternative translation strategies in more detail. 
In particular, further work on refinement of the micro/macro-step and whether correspondence of macro-steps can be relaxed; whether more complex refinement techniques could be supported (for example, using ranges in refinement annotations) would be useful; supporting/comparing alternative variations of semantics (by generating a different basis/scheduler for the translation).

For our interpretation of Statecharts in \mbox{\iUMLB}, we used the ``run-to-completion'' semantics of Statecharts.  In particular, we have carefully designed our translated model such that the semantics is captured as a generic abstract model, which is subsequently refined by the translation of the SCXML model.  An advantage of this approach is that we can easily adapt replace the basis model with other alternative semantics~\mbox{\cite{Eshuis_2009}} without changing the translation of the SCXML model. 

While Statecharts interpreted in iUML-B provide a way to incorporate refinement in an intuitive way, reversing this to \emph{discover} refinements holds promise. 
Checking a particular Statechart model for heirarchical structures that happen to follow the refinement proof obligations suggests an automatic way to accomplish abstract interpretation on an existing model.  
Such discovered abstraction/refinement relationships might improve the scalability of more complex Statechart models ``for free''.
%%% Local Variables: 
%%% mode: latex
%%% TeX-master: "../SCXMLREF.tex"
%%% End: 
