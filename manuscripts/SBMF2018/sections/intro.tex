% !TEX root = ../SCXMLREF.tex

\section{Introduction}
\label{sec:introduction}

Formal verification of high-consequence systems requires the analysis
of formal models that capture the properties and functionality of the
system of interest. Although high-consequence controls and systems are
designed to limit complexity, the requirements and consequent proof
obligations tend to increase the complexity of the formal verification.  
Proof obligations for such requirements can be made more tractable using
abstraction/refinement, providing a natural divide and conquer
strategy for controlling complexity.

\Statecharts~\cite{Harel} are often used for high-consequence controls
and other critical systems to provide an unambiguous, executable way
of specifying functional as well as safety, security, and reliability
properties.  While functional properties (usually) can be tested,
safety, security and reliability properties (usually) must be proved
formally.  Here we give a binding from \Statecharts to \EventB~\cite{abrial10:_model_event_b} so that
this type of reasoning can be carried out.  Moreover, hierarchical
encapsulation maps well onto \Statecharts in a way that is not very
different from previous work in \iUMLB~\cite{snook14:_b_statem,Snook2006,Snook12:FMCO}, a diagrammatic modelling notation for \EventB.
Binding \iUMLB to a UML~\cite{Rumbaugh2004} version of \Statecharts is natural and the
addition of run-to-completion semantics, expected by \Statechart
designers, is much of the contribution of this work.  Another
contribution is the augmentation of the textual and parse-able format
for \Statecharts, \SCXML to accommodate elements necessary to support formal
analysis. 

While \Statecharts and various semantic interpretations of
\Statecharts admit refinement reified as both hierarchical or parallel
composition (e.g. see Argos~\cite{Maraninchi91theargos}), here, as
previously~\cite{snook14:_b_statem}, we focus only on hierarchical
refinement, the form that \EventB natively admits.  Here we define
hierarchical composition to mean nesting new transition systems inside
previously pure states, and parallel composition to be the combination
in one machine of formerly separate transition systems.
A hierarchical development of a system model uses refinement
concepts to link the different levels of abstraction. Each subsequent
level increases model complexity by adding details in the form of
functionality and implementation method. As the model complexity
increases in each refinement level, tractability of the detailed model
can be improved by the use of a graphical representation, with rich
semantics that can support an infrastructure for formal verification.


The semantics adopted here adheres closely to UML \Statecharts~\cite{Alexandre} and is implemented in \iUMLB.
 Models described in \Statecharts are expressed in \SCXML and
 translated into \EventB logic which uses the \Rodin~\cite{abrial10:_rodin} for
 machine proofs.  UML \Statechart semantics are not the only formal
 semantics that can be bound to the \Statechart graphical
 language~\cite{Eshuis_2009}.  In Statecharts every triggering signal
 can cause transitions that emit other triggers in a cascade.
 Different semantic interpretations of Statecharts resolve these
 cascades differently.  Argos, for
 example, views cascading transitions as instantaneous and
 simultaneous rather than the queue-based semantics adopted here.
%
The \EventB modelling method provides the logic and refinement
theory required to formally analyse a system model.  The open-source
\Rodin provides support for \EventB
including automatic theorem provers.  \iUMLB
augments the \EventB language with a graphical interface including
state-machines.  

With suitable restrictions, \Statecharts already provide a sound,
intuitive, visual metaphor for refinement. Outfitted with a formal
semantics, this work borrows from well-used \Statechart practices in
digital design.  The goal of the present work is to provide usable,
well-founded tools that are familiar to designers of high-consequence
systems and yet provide the currently lacking formal guarantees needed
to ensure safety, security, and reliability.

We previously reported~\cite{Morris_2016} our early attempts to relate \Statecharts to \EventB. At that stage we had tried some aspects of the translation by using simplifications and we were beginning to gain insights into the problem, but had not arrived at the translation we now use.

The rest of the paper is structured as follows.  Section~\ref{sec:background} provides background information on \SCXML, \EventB, and \iUMLB.  Section~\ref{sec:secbot} presents our running example.  Section~\ref{sec:discussion} discusses the various challenges for introducing a refinement notion into \SCXML and demonstrates our approach.  Section~\ref{sec:extensions} shows our extensions to \SCXML which are necessary for reasoning about properties of \SCXML models.  In Section~\ref{sec:translation}, we illustrate our translation of \SCXML models into \EventB using the example introduced in Section~\ref{sec:secbot}.  Section~\ref{sec:example} shows how properties of the \SCXML models can be specified as invariants and verified in \EventB.  We summarise our contribution and conclude in Section~\ref{sec:conclusion}.

%%% Local Variables: 
%%% mode: latex
%%% TeX-master: "../SCXMLREF.tex"
%%% End: 
