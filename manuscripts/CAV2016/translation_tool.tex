% !TEX root = main.tex

\section{Translation Tool}

The iUML-B tooling is based on the Eclipse Modelling Framework (EMF). 
It is therefore beneficial to load the SCXML model into EMF so that 
our existing model to model transformation technology can be used to 
define the new SCXML to iUML-B translation. An EMF meta-model for SCXML 
is available from the Sirius \cite{??}
project. It supports SCXML functionality as well as providing generic model
loading facilities for new namespace extensions such as those we 
introduce in \ref{sect:extension}.

Hierarchical nested state charts are translated to similar corresponding 
state-machine structures in iUML-B. There are two alternative styles of 
Event-B representation for iUML-B state-machines.  Currently the state-variables 
style is adopted because it is simpler to translate from the SCXML model. 
However, the alternative state-enumeration style has benefits in user readability 
and may be supported in future. This would  require conventions regarding 
the name of the state-machine  to be adopted and used by the modeller in order 
to construct guards that refer to the current value of the state-machine.
SCXML features, such as initial states, entry/exit actions, and transition 
actions have corresponding similar features in iUML-B and their translation 
is relatively straightforward. Since SCXML `final' states, unlike iUML-B 
final states, are more akin to real states, their translation is less straightforward. 
An iUML-B state, final state and transition from the former to the latter are 
added to the corresponding iUML-B state-machine. The transition elaborates 
all Event-B events that are elaborated by transitions that exit the parent 
iUML-B state. 

\subsection{Refinement Levels}
An \emph{iumlb:refinement} attribute is used to indicate the first refinement 
level at which an element should be introduced in the generated iUML-B/Event-B model. 
In general, elements with no refinement attribute adopt that of their parent.  
However, for \emph{iscxml:state} elements, the refinement level refers to any state 
machines generated from the children nested in that state irrespective of whether 
those children specify a different refinement level. This is because generated 
iUML-B states cannot be added to an existing state-machine in later refinements.
For \emph{iumlb:invariants} the generated invariant is only  generated at the 
specified refinement level, not in  subsequent refinements. This is because Event-B 
invariants are visible through all subsequent refinements.

Note that our approach to refinement in SCXML largely restricts us to superposition 
refinement where entirely new details are introduced at each refinement level.  
It may be possible to support ranges in the refinement attribute enabling a model 
element to be replaced by some other element in a true data refinement. We plan to 
investigate this in future work, although, clearly, several coexisting alternative 
representations of the same concept may be problematic for the SCXML semantics.

\subsection{Constructing events elaborated by transitions}
The Event-B events that are elaborated by an iUML-B  transition are named as follows. 

\begin{enumerate}
\item If the transition has \emph{iumlb:label} attributes, events are generated 
and named according to the label attributes.
\item If the transition's source is an initial state at the outer state chart 
level the transition elaborates the special Event-B INITIALISATION event. 
\item If the transition's source is an initial state of a nested state chart 
the names of all the events that are associated with incoming transitions to 
the parent state are used.
\item If none of the above provide any labels, a default  `source\_target' format is used.
\end{enumerate}

Trigger events are deliberately not used for transition events because we 
want to keep them as a separate concept from transition firing in line 
with SCXML semantics.

\subsection{Data elements}
Data elements, collated in \emph{iscxml:datamodel} elements, model ancillary 
variables in the way usual SCXML style. Data elements are translated to 
Event-B variables of type given in an \emph{iumlb:type} attribute translated 
into an Event-B subset invariant.  The \emph{iscxml:id} attribute of the 
\emph{iscxml:data} element is interpreted as the name of the variable and 
the value is used as the right hand side of an assignment action to initialise 
the variable.  Some syntax conversion is performed to convert the predicate from 
SCXML format into the Event-B mathematical language. The variable is introduced 
at the same refinement level as the parent element that contains it.


%------------------------------------------------------------------------------
